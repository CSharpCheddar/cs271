\documentclass[12pt]{article}

\usepackage{listings}
\usepackage{color}
\usepackage{amsmath}

\definecolor{dkgreen}{rgb}{0,0.6,0}
\definecolor{gray}{rgb}{0.5,0.5,0.5}
\definecolor{mauve}{rgb}{0.58,0,0.82}

\lstset{frame=tb,
  language=Java,
  aboveskip=3mm,
  belowskip=3mm,
  showstringspaces=false,
  columns=flexible,
  basicstyle={\small\ttfamily},
  numbers=none,
  numberstyle=\tiny\color{gray},
  keywordstyle=\color{blue},
  commentstyle=\color{dkgreen},
  stringstyle=\color{mauve},
  breaklines=true,
  breakatwhitespace=true
  tabsize=3
}

\begin{document} 
\title{Mini Assignment 2}
\author{Due: 11:20am on February $12^{th}$, 2019}
\date{20 points}
\maketitle

Give the worst-case analysis for the following functions in terms of big-$\Theta$. You should submit your pdf solution using \LaTeX\ if possible.
    \begin{enumerate}
    	\item 
\begin{lstlisting}
for(i = 0; i < 3; i++){
    for(j = 0; j < 10; j++){
        print i+j;
    }
}
\end{lstlisting}
There are two nested for loops which run a finite amount of iterations that each add 1 to their variable after completion, and there is one constant time operation performed in the second for loop, therefore, this function has a run time of $\boxed{\Theta(1)}$.
    	\item 
\begin{lstlisting}
//n and m are some positive integers
for(i = 0; i < n; i++){
    for(j = 0; j < m; j++){
        print i+j;
    }
}
\end{lstlisting}
There are two nested for loops similar to the ones above with one constant time operation in the middle, therefore, this function has a run time of $\boxed{\Theta(nm)}$.
\newpage
    	\item 
\begin{lstlisting}
//n and m are some positive integers
for(i = 0; i < n; i++){
    for(j = 0; j < m; j++){
        for(int k = 1; k < 1000; k *= 2){
        	print i+j+k;
        }
    }
}
\end{lstlisting}
Although there are 3 nested for loops, the innermost one runs a constant number of times and can be ignored. This reduces the problem to one similar to the one above. There are two nested for loops that depend on two variables $n$ and $m$, therefore, the run time of this function is $\boxed{\Theta(nm)}$.
    	\item 
\begin{lstlisting}
//n and m are some positive integers
for(i = 0; i < n - 10; i++){
    for(j = 0; j < m/2; j++){
        print i+j;
    }
}
\end{lstlisting}
Similar to the one above, there are two nested for loops each dependent on the two variable integers $n$ and $m$, however it differs from above by the constants that are appended in each loop. Since these are constants however, they can be ignored and the function would, therefore, have a run time of $\boxed{\Theta(nm)}$.
\newpage
    	\item 
\begin{lstlisting}
//n and m are some positive integers
for(i = 0; i < n; i++){
    print i;
}
//n and m are some integers
for(j = 1; j < m; j *= 2){
    print j;
}
\end{lstlisting}
These for loops are not nested, and therefore must be con considered separately. The top for loop looks similar to the ones above and would have a run time of about $n$. The bottom for loop, however, would have a run time of $\log(n)$ because the value of $j$ grows much faster than $i$ in the previous loop, therefore, the function would have an overall big theta value of $\boxed{\Theta(n)}$.
    \end{enumerate}

\end{document}