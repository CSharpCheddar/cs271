\documentclass[12pt]{article}

\usepackage{amsmath}
\usepackage{listings}
\usepackage{color}
\usepackage{hyperref}
\usepackage{tikz}
\usetikzlibrary{trees}

\definecolor{dkgreen}{rgb}{0,0.6,0}
\definecolor{gray}{rgb}{0.5,0.5,0.5}
\definecolor{mauve}{rgb}{0.58,0,0.82}

\lstset{frame=tb,
  language=Java,
  aboveskip=3mm,
  belowskip=3mm,
  showstringspaces=false,
  columns=flexible,
  basicstyle={\small\ttfamily},
  numbers=none,
  numberstyle=\tiny\color{gray},
  keywordstyle=\color{blue},
  commentstyle=\color{dkgreen},
  stringstyle=\color{mauve},
  breaklines=true,
  breakatwhitespace=true
  tabsize=3
}

\begin{document} 
\title{Mini Assignment 18}
\author{Due: 11:20am on May $8^{th}$, 2019}
\date{20 points}
\maketitle

Insert the following edges into a directed graph: (A,B,10), (A,C,10), (B,D,4), (B,C,3), (B,E,4), (B,F,2), (C,D,3), (C,E,3), (C,F,10), (D,E,2), (D,G,10), (E,G,5), (F,G,5), (F,D,4). Assume that A is your source and G is your sink. What is the maximum flow in this graph? You are encouraged to show your residual and flow graphs for partial credit. However, only the maximum flow value is needed.
\begin{center}
	$\boxed{20}$
\end{center}

\end{document}