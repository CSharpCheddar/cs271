\documentclass[12pt]{article}

\usepackage{tikz}
\usetikzlibrary{trees}

\begin{document} 
\title{Mini 11}
\author{Martin Mueller}
\date{Due: 11:20am on March $18^{th}$, 2019}
\maketitle

\begin{enumerate}
        \item \textbf{Assume we have a tree of height 10. We insert a node at the $11^{th}$ level. Is it possible for only the root node to be unbalanced? A yes or no answer is sufficient, no justification is needed.}

		\begin{center}
			No
		\end{center}		        
        
		\item \textbf{Can inserting a node cause more than 1 rotation (single or double)? Provide some justification to back up your answer.}
		
		\begin{center}
			Yes because each time a node is balanced, the pointers that are changing could cause a chain reaction that would cause each node above it to become unbalanced. If you think about the tree as a rigid structure, (as we usually do), we can just imagine that some elements are pulled up, while others are pushed down (which would require a chain of rotations that go up the tree).
		\end{center}				
		
		\item \textbf{Can removing a node cause more than 1 rotation (single or double)? Provide some justification to back up your answer.}
		
		\begin{center}
			Yes. Just as I stated above, moving around the pointers of a group of nodes can create a chain reaction that goes back up the tree.
		\end{center}			
		
		\pagebreak		
		
        \item \textbf{What does the AVL tree look like if we add 1 through 6 into the tree in that order?}
        
        \begin{center}
        	\begin{tikzpicture}[level distance=1.5cm,
		  	level 1/.style={sibling distance=3cm},
		  	level 2/.style={sibling distance=1.5cm}]
		  		\node {4}
		    		child {node {2}
  	 	 	  		child {node {1}}
   	 		  		child {node {3}}
    				}
   		 			child {node {5}
   	 				child {}
      				child {node {6}}
  	  				};
			\end{tikzpicture}
			
			\textit{Note: The extra branch off of node 5 is there so 6 is more clearly shown as the right child 5.}
        \end{center}
    \end{enumerate}
\end{document}